\PassOptionsToPackage{unicode=true}{hyperref} % options for packages loaded elsewhere
\PassOptionsToPackage{hyphens}{url}
%
\documentclass[]{article}
\usepackage{lmodern}
\usepackage{amssymb,amsmath}
\usepackage{ifxetex,ifluatex}
\usepackage{fixltx2e} % provides \textsubscript
\ifnum 0\ifxetex 1\fi\ifluatex 1\fi=0 % if pdftex
  \usepackage[T1]{fontenc}
  \usepackage[utf8]{inputenc}
  \usepackage{textcomp} % provides euro and other symbols
\else % if luatex or xelatex
  \usepackage{unicode-math}
  \defaultfontfeatures{Ligatures=TeX,Scale=MatchLowercase}
\fi
% use upquote if available, for straight quotes in verbatim environments
\IfFileExists{upquote.sty}{\usepackage{upquote}}{}
% use microtype if available
\IfFileExists{microtype.sty}{%
\usepackage[]{microtype}
\UseMicrotypeSet[protrusion]{basicmath} % disable protrusion for tt fonts
}{}
\IfFileExists{parskip.sty}{%
\usepackage{parskip}
}{% else
\setlength{\parindent}{0pt}
\setlength{\parskip}{6pt plus 2pt minus 1pt}
}
\usepackage{hyperref}
\hypersetup{
            pdftitle={Incentivos al Retiro},
            pdfborder={0 0 0},
            breaklinks=true}
\urlstyle{same}  % don't use monospace font for urls
\usepackage[margin=1in]{geometry}
\usepackage{graphicx,grffile}
\makeatletter
\def\maxwidth{\ifdim\Gin@nat@width>\linewidth\linewidth\else\Gin@nat@width\fi}
\def\maxheight{\ifdim\Gin@nat@height>\textheight\textheight\else\Gin@nat@height\fi}
\makeatother
% Scale images if necessary, so that they will not overflow the page
% margins by default, and it is still possible to overwrite the defaults
% using explicit options in \includegraphics[width, height, ...]{}
\setkeys{Gin}{width=\maxwidth,height=\maxheight,keepaspectratio}
\setlength{\emergencystretch}{3em}  % prevent overfull lines
\providecommand{\tightlist}{%
  \setlength{\itemsep}{0pt}\setlength{\parskip}{0pt}}
\setcounter{secnumdepth}{0}
% Redefines (sub)paragraphs to behave more like sections
\ifx\paragraph\undefined\else
\let\oldparagraph\paragraph
\renewcommand{\paragraph}[1]{\oldparagraph{#1}\mbox{}}
\fi
\ifx\subparagraph\undefined\else
\let\oldsubparagraph\subparagraph
\renewcommand{\subparagraph}[1]{\oldsubparagraph{#1}\mbox{}}
\fi

% set default figure placement to htbp
\makeatletter
\def\fps@figure{htbp}
\makeatother


\title{Incentivos al Retiro}
\author{}
\date{\vspace{-2.5em}}

\begin{document}
\maketitle

{
\setcounter{tocdepth}{3}
\tableofcontents
}
\hypertarget{introducciuxf3n}{%
\section{Introducción}\label{introducciuxf3n}}

Importancia de alargar vida laboral. OCDE 2011.

\hypertarget{marco-conceptual}{%
\section{Marco Conceptual}\label{marco-conceptual}}

\hypertarget{modelo-de-oferta-de-trabajo}{%
\subsection{Modelo de oferta de
trabajo}\label{modelo-de-oferta-de-trabajo}}

\hypertarget{otros-canales}{%
\subsection{Otros Canales}\label{otros-canales}}

\hypertarget{paruxe1metros-de-la-ss}{%
\subsection{Parámetros de la SS}\label{paruxe1metros-de-la-ss}}

\hypertarget{incentivos-de-la-ss}{%
\subsection{Incentivos de la SS}\label{incentivos-de-la-ss}}

\hypertarget{primeros-trabajos-empuxedricos}{%
\subsection{Primeros Trabajos
empíricos}\label{primeros-trabajos-empuxedricos}}

\hypertarget{poluxedticas-de-las-uxfaltimas-duxe9cadas}{%
\subsection{Políticas de las últimas
décadas}\label{poluxedticas-de-las-uxfaltimas-duxe9cadas}}

\hypertarget{evaluaciuxf3n-de-estas-poluxedticas}{%
\subsection{Evaluación de estas
políticas}\label{evaluaciuxf3n-de-estas-poluxedticas}}

\hypertarget{la-edad-muxednima-de-retiro}{%
\subsection{La Edad Mínima de
Retiro}\label{la-edad-muxednima-de-retiro}}

Entre 1980 y 2018, la EMR subió en Bélgica, Dinamarca, Francia,
Alemania, Japón, Holanda, España, Suecia y el Reino Unido
(({\textbf{???}})). Como veremos más adelante, las variaciones de este
parámetro de política pueden usarse para estimar la respuesta en los
trabajadores con precisión.

\hypertarget{la-edad-normal-de-retiro}{%
\subsection{La Edad Normal de Retiro}\label{la-edad-normal-de-retiro}}

La ENR también subió generalizadamente en los países de la OCDE.

\hypertarget{ajustes-actuariales}{%
\subsection{Ajustes actuariales}\label{ajustes-actuariales}}

Los ajustes actuariales penalizan a los trabajadores que se retiran
antes de la ENR. Dado que al postergar un año el retiro, el trabajador
contribuye un año más a la seguridad social y recibe un año menos de
prestaciones, es usual que el valor de estas prestaciones sea mayor
cuanto más postergue el trabajaro el acceso a las mismas.

En este trabajo presentamos la literatura empírica relacionada al
impacto de estos instrumentos sobre la participación laboral y el uso de
otros programas de la seguridad social (seguro de desempleo, invalidez,
enfermedad) de la población afectada.

\hypertarget{algunos-conceptos}{%
\section{Algunos conceptos}\label{algunos-conceptos}}

Caída de la participación masculina en las últimas décadas del siglo XX.
Citar alguno gráfico. Generosidad de los sistemas de seguridad social y
las posibilidades de retiro temprano esten atrás de eso.

Literatura original de incentivos al retiro. Diamond y Gruber 1997,
breve resúmen de la literatura desde los 80s.

Muchos cambios en los sistemas previsionales desde los años 80. Si bien
es difícil condensar todo en un solo indicador, la orientación general
de los cambios es hacia fortalecer los incentivos a trabajar.
({\textbf{???}})

\hypertarget{riqueza-de-la-seguridad-social-ssw}{%
\subsection{Riqueza de la Seguridad Social
(SSW)}\label{riqueza-de-la-seguridad-social-ssw}}

Para medir estos incentivos empíricamente, se calcula la riqueza de la
seguridad social (SSW). Esta medida compara los ingresos que el
trabajador recibe por trabajar con los cambios en el flujo de
prestaciones que le paga el sistema de jubilaciones.

La Riqueza de la Seguridad Social (SSW) para una edad de retiro \(h\) es
el valor actualizado de los beneficios recibidos entre \(h + 1\) y la
fecha de muerte \(S\).

\[ SSW_{h} = \sum_{s=h+1}^{\S} \rho_{s} B_{s}(h)\]

Para actualizar los beneficios se utiliza el factor de descuento
\(\rho\), que depende de la probabilidad de sobrevivencia y el factor de
descuento intertemporal.

Si seguir trabajando un año reduce el flujo de pagos que el individuo
recibe de la seguridad social, esta diferencia actúa como un impuesto a
seguir trabajando. Este desincentivo a seguir en la fuerza de trabajo
una vez alcanzadas las condiciones para acceder a una jubilación
funciona como un impuesto sobre el trabajo.

\hypertarget{devengamiento-de-la-riqueza-e-impuesto-impluxedcio-al-trabajo}{%
\subsection{Devengamiento de la Riqueza e impuesto implício al
trabajo}\label{devengamiento-de-la-riqueza-e-impuesto-impluxedcio-al-trabajo}}

El devengamiento de \(SSW\) es la medida más simple de incentivo a
trabajar, consiste en calcular el cambio en SSW por permanecer en la
fuerza de trabajo un año más:

\[ SSA_{a} = SSW_{a+1} - SSW_{a}\]

El impuesto implícito a seguir trabajando se calcula como

\[ \tau_{a} = \frac{-SSA_{a}}{W_{a+1}}\]

El principal problema que tiene es que esta medida solo mira un período
para adelante, pero la SSA no se incrementa monotónicamente, sino que
puede tener saltos.

Eso implica que para un trabajador la SSA de un año puede ser baja, pero
la de 4 años para adelante tener un salto brusco, por lo que el
trabajador sigue trabajando teniendo ese salto en cuenta.

\hypertarget{peak-value}{%
\subsubsection{Peak Value}\label{peak-value}}

Para resolver esta limitación, se propone la medida de \emph{Peak
Value}, que calcula \(PV_{a} = max_{h}(SSW_{h}-SSW_{a})\), o sea, la
diferencia máxima entre la riqueza de la seguridad social de retirarse
hoy o retirarse en otra fecha.

\hypertarget{option-value}{%
\subsubsection{Option Value}\label{option-value}}

Ver Gustman

\hypertarget{alemania}{%
\subsubsection{Alemania}\label{alemania}}

Este trabajo analiza el efecto de una reforma jubilatoria en Alemania
que introduce una reducción de las prestaciones para los trabajadores
que se retiran antes de la edad normal.

Estas reducciones pueden ir entre 0.3\% y 18\% según la cohorte a la que
pertenece el individuo y el momento en que solicita la jubilación.

El objetivo es medir si los trabajadores difieren la solicitud de la
jubilación y/o la salida del mercado laboral.

No es lo mismo, y si afecta uno diferente de la otra puede tener
implicancias sobre el impacto fiscal.

Encuentran que la reforma pospone el acceso a los beneficios
jubilatorios en 14 meses. Las mujeres y los trabajadores con menores
jubilaciones difieren en mayor medida.

La salida del mercado laboral se difiere en 10 meses. Este efecto es
mayor en los hombres y en trabajadores con jubilaciones mayores.

También encuentran que los trabajadores reaccionan poco a los cambios en
la SSW, y fuertemente a los ACCRUALS.

\hypertarget{austria}{%
\subsubsection{Austria}\label{austria}}

Este trabajo analiza una serie de cambios paraméticos en el sistema
jubilatorio austríaco en 2000 y 2004. Estos cambios implicaron un
aumento de la edad mínima de retiro (EMR). La introducción gradual de
este cambio permite identificar los efectos con precisión.

Se estima que el aumento en la ERA implicón un aumento de 0.4 años en la
salida del mercado laboral y 0.5 años en la edad de solicitud de la
jubilación.

La magnitud similar de este cambio implica que no se registraron
aumentos importantes en el uso de otros programas que pueden servir como
salidas alternativas del mercado de trabajo. El principal mecanismo de
ajuste es la mayor permanencia de los trabajadores en sus puestos de
trabajo.

\hypertarget{reino-unido}{%
\subsection{Reino Unido}\label{reino-unido}}

This paper uses evidence on labor market behavior in the UK between 2010
and 2014 to examine what impact increasing the ERA from 60 to 62 has had
on the economic activity of the affected cohorts of women.

\hypertarget{literatura}{%
\subsubsection{Literatura}\label{literatura}}

Blundell and Emmerson (2007) estimate that a three-year increase in the
ERA for both men and women (and assuming that defined benefit
occupational pension schemes respond with a three-year increase in their
normal pension ages as well) would increase retirement ages by between
0.4 and 1.8 years, depending on the specification used.

However, while the effects estimated in these ex ante simulations were
quite large, if anything the results of ex post evaluations suggests
even larger effects.

Although almost all of these have focused on changes to normal, rather
than early, retirement ages. The two major exceptions are Staubli and
Zweimüller (2013) and Atalay and Barrett (2015), who examine the effect
of changes in ERAs. They find that a one year increase in the ERA led to
an increase in employment rates of 9.75 percentage points for affected
men and by 11 percentage points for affected women, with increases in
unemployment rates of a similar size.

Manoli and Weber (2016) study the same Austrian reforms and find large
delays in job exits and pension claiming caused by the increase in the
ERA. However, the Austrian state pension system is different from the UK
(and a number of other coun- tries' systems) in several important ways.
First, in the Austrian system, individuals' pension benefits are
completely withdrawn if their earnings exceed around \$500 a month.
Second, although the Austrian system provides some increase in pension
income for delayed drawing, this is done at a less than actuarially fair
rate.

Third, the Austrian state pension provides a very high level of earnings
replacement (according to Staubli and Zweimüller (2013), the average net
replacement rate of pre- retirement earnings is 75\%); public pensions,
therefore, provide the main source of income for most pensioners in
Austria.

Atalay and Barrett (2015) examine the effect of an increase in the
earliest age at which women can access the Australian Age Pension. They
find, using cross-sectional survey data, that a one year increase in the
eligibility age induced a 12--19 percentage point increase in fe- male
labor supply.

A number of papers have conducted ex ante simulation of such reforms
using out-of-sample predictions, which suggested quite large equilibrium
effects in many countries. Ex post evaluations of changes to normal and
early retirement ages have tended to find, if any- thing, larger effects
than were suggested by the ex ante simulations.

\hypertarget{la-poluxedtica}{%
\subsubsection{La política}\label{la-poluxedtica}}

En 1995, UK aprobó un aumento de la EDA para las mujeres. La ERA subiría
gradualmente de 60 a 65 años entre 2010 y 2020.

\hypertarget{los-resultados}{%
\subsubsection{Los resultados}\label{los-resultados}}

Identifican el impacto de la ERA comparando cohortes que enfrentan
diferentes ERAs.

Las tasas de empleo de las mujeres de 60 y 61 años aumentó 6.3 puntos
porcentuales cuando la ERA subió de 60 a 62 años. Este efecto es implica
un aumento de dos meses en la edad de retiro promedio, e implica que 3/4
de los retiros en exceso a los 60 se explican debido a la ERA.

Estos resultados también sugieren un fuerte efecto señalización de la
ERA. En UK hay incentivos financieros reducidos a retirarse en la ERA.
We find the policy has also led to a 1.2 percentage point increase in
the fraction of women who are unemployed and actively seeking work at
ages 60 and 61. We also find a 4.0 percentage point increase in the
proportion of women reporting themselves to be economically inactive due
to sickness or disability.

\hypertarget{lecciones-de-poluxedtica}{%
\subsection{Lecciones de política}\label{lecciones-de-poluxedtica}}

\begin{itemize}
\tightlist
\item
  La edad mínima de jubilación es un instrumento altamente eficaz para
  prolongar la vida laboral de los mayores. Efecto señalización es parte
  del impacto.
\item
  Los incentivos actuariales también funcionan.
\item
  Jubilaciones mínimas pueden ser un incentivo a salir temprano del
  mercado a pesar de los ajustes actuariales.
\item
  Ojo con los otros programas.
\item
  Salud y educación, restricciones en las oportunidades de empleo.
\item
  Hacia donde va la agenda de investigación.
\end{itemize}

\hypertarget{glosario}{%
\section{Glosario}\label{glosario}}

\begin{description}
\item[Edad Efectiva de Retiro]
Es la edad a la que un individuo empieza a recibir prestaciones del
sistema de seguridad social.
\item[Edad Mínima de Retiro]
Es la menor edad a la que un trabajador puede aplicar a un programa de
seguridad social. Es usual que en estos casos las prestaciones se vean
reducidas frente a las recibidas en la edad normal de retiro.
\item[Retiro Temprano]
Es la práctica de acogerse a los beneficios de la seguirdad social antes
de la edad normal de retiro.
\item[Prueba de Ingresos]
Es un límite a los ingresos que puede tener alguien que recibe
prestaciones del sistema de segurida social.
\item[Impuestos Implícitos]
Es el impuesto implícito que enfrenta un trabajador cuando, pudiendo
recibir una prestación del sistema de seguirdad social, decide seguir
trabajando y sus beneficios futuros no son compensados.
\end{description}

\hypertarget{referencias}{%
\section{Referencias}\label{referencias}}

\end{document}
