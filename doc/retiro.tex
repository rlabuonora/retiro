\PassOptionsToPackage{unicode=true}{hyperref} % options for packages loaded elsewhere
\PassOptionsToPackage{hyphens}{url}
%
\documentclass[]{article}
\usepackage{lmodern}
\usepackage{amssymb,amsmath}
\usepackage{ifxetex,ifluatex}
\usepackage{fixltx2e} % provides \textsubscript
\ifnum 0\ifxetex 1\fi\ifluatex 1\fi=0 % if pdftex
  \usepackage[T1]{fontenc}
  \usepackage[utf8]{inputenc}
  \usepackage{textcomp} % provides euro and other symbols
\else % if luatex or xelatex
  \usepackage{unicode-math}
  \defaultfontfeatures{Ligatures=TeX,Scale=MatchLowercase}
\fi
% use upquote if available, for straight quotes in verbatim environments
\IfFileExists{upquote.sty}{\usepackage{upquote}}{}
% use microtype if available
\IfFileExists{microtype.sty}{%
\usepackage[]{microtype}
\UseMicrotypeSet[protrusion]{basicmath} % disable protrusion for tt fonts
}{}
\IfFileExists{parskip.sty}{%
\usepackage{parskip}
}{% else
\setlength{\parindent}{0pt}
\setlength{\parskip}{6pt plus 2pt minus 1pt}
}
\usepackage{hyperref}
\hypersetup{
            pdftitle={Incentivos al Retiro},
            pdfborder={0 0 0},
            breaklinks=true}
\urlstyle{same}  % don't use monospace font for urls
\usepackage[margin=1in]{geometry}
\usepackage{graphicx,grffile}
\makeatletter
\def\maxwidth{\ifdim\Gin@nat@width>\linewidth\linewidth\else\Gin@nat@width\fi}
\def\maxheight{\ifdim\Gin@nat@height>\textheight\textheight\else\Gin@nat@height\fi}
\makeatother
% Scale images if necessary, so that they will not overflow the page
% margins by default, and it is still possible to overwrite the defaults
% using explicit options in \includegraphics[width, height, ...]{}
\setkeys{Gin}{width=\maxwidth,height=\maxheight,keepaspectratio}
\setlength{\emergencystretch}{3em}  % prevent overfull lines
\providecommand{\tightlist}{%
  \setlength{\itemsep}{0pt}\setlength{\parskip}{0pt}}
\setcounter{secnumdepth}{0}
% Redefines (sub)paragraphs to behave more like sections
\ifx\paragraph\undefined\else
\let\oldparagraph\paragraph
\renewcommand{\paragraph}[1]{\oldparagraph{#1}\mbox{}}
\fi
\ifx\subparagraph\undefined\else
\let\oldsubparagraph\subparagraph
\renewcommand{\subparagraph}[1]{\oldsubparagraph{#1}\mbox{}}
\fi

% set default figure placement to htbp
\makeatletter
\def\fps@figure{htbp}
\makeatother


\title{Incentivos al Retiro}
\author{}
\date{\vspace{-2.5em}}

\begin{document}
\maketitle

{
\setcounter{tocdepth}{3}
\tableofcontents
}
\hypertarget{introducciuxf3n}{%
\section{Introducción}\label{introducciuxf3n}}

Importancia de alargar vida laboral. OCDE 2011.

\hypertarget{teoruxeda}{%
\subsection{Teoría}\label{teoruxeda}}

En este contexto, dos instrumentos de política clave que han sido
utilizados por los gobiernos de muchos países han sido la edad mínima de
retiro (EMR), la edad normal de retiro (ENR) y los ajustes actuariales
que penalizan a los trabajadores que se retiran antes de la ENR.

Uno de los hechos estilizados es que una gran proporción de los
trabajadores se retira en la EMR. Citar a ({\textbf{???}}). Para
Uruguay, ver ({\textbf{???}}). En este contexto, aumentar la edad mínima
a la que los trabajadores son elegibles para cobrar jubilaciones surge
como una medida obvia para mejorar la sostenibilidad de los sistemas.

Los ajustes actuariales a las tasas de remplazo también se usan como
incentivos para que los trabajadores difieran su retiro del mercado
laboral y/o accedan a jubilaciones.

Desde fines del siglo XX muchos países han implentado cambios en estos
parámetros para mejorar la sostenibilidad de los sistemas jubilatorios.
En este trabajo presentamos la evidencia empírica disponible sobre los
efectos de estas reformas.

Cribb, Emmerson, and Tetlow (2016) identifican 4 mecanismos mediante los
que la ERA afecta la actividad económica de los trabajadores:

\begin{itemize}
\tightlist
\item
  Primero, cambia las tasas de impuestos relativas al trabajo.
\item
  Segundo, reduce el ingreso y la \emph{lifetime wealth} de los
  individuos porque reduce los ingresos que van a recibir de la
  seguridad social.
\item
  Individuos con restricciones de liquidez pueden verse obligados a
  seguir trabajando para sostener su consumo.
\item
  La edad mínima de retiro puede ser una señal para los trabajadores, de
  que la edad marcada es la edad \emph{correcta} para jubilarse.
\end{itemize}

({\textbf{???}}) cita el modelo de Stock y Wise:

Consider that an individual takes the decision regarding the optimal
date of retirement entry according to the Option Value Model of Stock
and Wise (1990), i.e.~she compares all possible future streams of
utility from income and leisure and delays retirement entry if she can
thereby receive a higher stream of utility.

In this framework, retiring at any early date s instead of any later
dater t has four effects: (i) it lowers utility due to a loss of wage
earnings during the period between s and t, (ii) it increases utility
from leisure during s and t, (iii) it lengthens the period of benefit
receipt, thereby raising utility according to the amount of benefits
that are paid between s and t, and (iv) it changes the expected present
value of future benefits during the remaining lifetime according to the
pension formula. Hence, retiring at date s instead of t will be
preferred only if the loss in wage income is at least outweighed by (a)
the utility from leisure between s and t, (b) the retirement income
received between s and t, plus (b) the difference in the present values
of future income if the individual retires at date s instead of date t.

\hypertarget{la-edad-muxednima-de-retiro}{%
\subsection{La Edad Mínima de
Retiro}\label{la-edad-muxednima-de-retiro}}

Entre 1980 y 2018, la EMR subió en Bélgica, Dinamarca, Francia,
Alemania, Japón, Holanda, España, Suecia y el Reino Unido
(({\textbf{???}})). Como veremos más adelante, las variaciones de este
parámetro de política pueden usarse para estimar la respuesta en los
trabajadores con precisión.

\hypertarget{la-edad-normal-de-retiro}{%
\subsection{La Edad Normal de Retiro}\label{la-edad-normal-de-retiro}}

La ENR también subió generalizadamente en los países de la OCDE.

\hypertarget{ajustes-actuariales}{%
\subsection{Ajustes actuariales}\label{ajustes-actuariales}}

Los ajustes actuariales penalizan a los trabajadores que se retiran
antes de la ENR. Dado que al postergar un año el retiro, el trabajador
contribuye un año más a la seguridad social y recibe un año menos de
prestaciones, es usual que el valor de estas prestaciones sea mayor
cuanto más postergue el trabajaro el acceso a las mismas.

Explicar lo del earnings test y el país que no tiene ()

Most countries feature an earnings test at ages before the SEA. This
forces individuals to stop working when they want to receive social
security benefits, as benefits are taxed, often dollar-for-dollar,
against earnings (although a small amount of earnings may be allowed
without taxation). The decision to claim benefits and the decision to
exit the labor force, which are independent decisions from an
individual's point of view, are thus intrinsically combined in these
countries; this helps to explain why the word ``retirement'' means both
decisions in these countries. An earnings test is currently in place
before the SEA in Belgium, Canada, Denmark, Germany, Japan, Spain, and
the UK; only France eliminated its earnings test during the period we
examine.

En este trabajo presentamos la literatura empírica relacionada al
impacto de estos instrumentos sobre la participación laboral y el uso de
otros programas de la seguridad social (seguro de desempleo, invalidez,
enfermedad) de la población afectada.

Esta literatura tiene implicancias a la hora de evaluar como
implementarlas en Uruguay.

Las evaluaciones ex-post de las reformas implementadas en Austria,
Australia, Alemania y Francia implican que un aumento de 1 o 2 años en
la edad mínima de retiro tiene un impacto de entre 6.5 y 20 puntos
porcentuales en la probabilidad de que las personas afectadas se
encuentren trabajando. Ver!

Por otro lado, Mastrobouni encuentra que

\hypertarget{algunos-conceptos}{%
\section{Algunos conceptos}\label{algunos-conceptos}}

Caída de la participación masculina en las últimas décadas del siglo XX.
Citar alguno gráfico. Generosidad de los sistemas de seguridad social y
las posibilidades de retiro temprano esten atrás de eso.

Literatura original de incentivos al retiro. Diamond y Gruber 1997,
breve resúmen de la literatura desde los 80s.

Muchos cambios en los sistemas previsionales desde los años 80. Si bien
es difícil condensar todo en un solo indicador, la orientación general
de los cambios es hacia fortalecer los incentivos a trabajar.
({\textbf{???}})

\hypertarget{riqueza-de-la-seguridad-social-ssw}{%
\subsection{Riqueza de la Seguridad Social
(SSW)}\label{riqueza-de-la-seguridad-social-ssw}}

Para medir estos incentivos empíricamente, se calcula la riqueza de la
seguridad social (SSW). Esta medida compara los ingresos que el
trabajador recibe por trabajar con los cambios en el flujo de
prestaciones que le paga el sistema de jubilaciones.

La Riqueza de la Seguridad Social (SSW) para una edad de retiro \(h\) es
el valor actualizado de los beneficios recibidos entre \(h + 1\) y la
fecha de muerte \(S\).

\[ SSW_{h} = \sum_{s=h+1}^{\S} \rho_{s} B_{s}(h)\]

Para actualizar los beneficios se utiliza el factor de descuento
\(\rho\), que depende de la probabilidad de sobrevivencia y el factor de
descuento intertemporal.

Si seguir trabajando un año reduce el flujo de pagos que el individuo
recibe de la seguridad social, esta diferencia actúa como un impuesto a
seguir trabajando. Este desincentivo a seguir en la fuerza de trabajo
una vez alcanzadas las condiciones para acceder a una jubilación
funciona como un impuesto sobre el trabajo.

\hypertarget{devengamiento-de-la-riqueza-e-impuesto-impluxedcio-al-trabajo}{%
\subsection{Devengamiento de la Riqueza e impuesto implício al
trabajo}\label{devengamiento-de-la-riqueza-e-impuesto-impluxedcio-al-trabajo}}

El devengamiento de \(SSW\) es la medida más simple de incentivo a
trabajar, consiste en calcular el cambio en SSW por permanecer en la
fuerza de trabajo un año más:

\[ SSA_{a} = SSW_{a+1} - SSW_{a}\]

El impuesto implícito a seguir trabajando se calcula como

\[ \tau_{a} = \frac{-SSA_{a}}{W_{a+1}}\]

El principal problema que tiene es que esta medida solo mira un período
para adelante, pero la SSA no se incrementa monotónicamente, sino que
puede tener saltos.

Eso implica que para un trabajador la SSA de un año puede ser baja, pero
la de 4 años para adelante tener un salto brusco, por lo que el
trabajador sigue trabajando teniendo ese salto en cuenta.

\hypertarget{peak-value}{%
\subsubsection{Peak Value}\label{peak-value}}

Para resolver esta limitación, se propone la medida de \emph{Peak
Value}, que calcula \(PV_{a} = max_{h}(SSW_{h}-SSW_{a})\), o sea, la
diferencia máxima entre la riqueza de la seguridad social de retirarse
hoy o retirarse en otra fecha.

\hypertarget{option-value}{%
\subsubsection{Option Value}\label{option-value}}

Cambios en la edad de elegibilidad. Teoría. Indicadores de incentivos.
Stock y Wise. (Ver versión de Aguila)

(Rabaté)

Due to credit constraints, individuals may not be able to smooth their
consumption without delaying their exit from the labor force. It may
also generate a negative wealth effect, as the maximum duration of
benefits receipt decreases.

Ver Gustman

\hypertarget{resultados}{%
\section{Resultados}\label{resultados}}

\hypertarget{microestimation-alemania-inglaterra}{%
\subsection{Microestimation (Alemania,
Inglaterra)}\label{microestimation-alemania-inglaterra}}

\hypertarget{muxe9jico-chile-brasil}{%
\subsection{Méjico, Chile, Brasil}\label{muxe9jico-chile-brasil}}

\hypertarget{evidencia-reciente}{%
\subsection{Evidencia Reciente}\label{evidencia-reciente}}

Problemas de la evidencia anterior. Ex-ante, basada en simulaciones. La
evidencia más reciente consiste en evaluar los efectos observados de
medidas efectivamente implementadas.

Los estudio ex-ante estiman menores impactos de la seguridad social en
la participación laboral. Principales razones que atenúan las
estimaciones:

\begin{itemize}
\tightlist
\item
  Primero, no capturan efectos de posibles normas sociales asociadas a
  Edades Mínimas.
\item
  Segundo, los cálcuos de las medidas de incentivos tienen error de
  medida por falta de información (estructura familiar, etc.).
\item
  Tercero, las estimaciones tienen sesgos de endogeneidad por las
  correlaciones entre las historias laborales, las preferencias por el
  trabajo y los incentivos al retiro.
\end{itemize}

En general, la evidencia se basa en diseños cuasi-experimentales. Estos
diseños implican encontrar grupos de control que no se vean afectados
por los cambios introducidos.

Esto permite aislar el impacto de los cambios analizados con precisión.

Otra técnica usada en los trabajos de última generación son las de
Regression Kinks.

\hypertarget{alemania}{%
\subsubsection{Alemania}\label{alemania}}

Este trabajo analiza el efecto de una reforma jubilatoria en Alemania
que introduce una reducción de las prestaciones para los trabajadores
que se retiran antes de la edad normal.

Estas reducciones pueden ir entre 0.3\% y 18\% según la cohorte a la que
pertenece el individuo y el momento en que solicita la jubilación.

El objetivo es medir si los trabajadores difieren la solicitud de la
jubilación y/o la salida del mercado laboral.

No es lo mismo, y si afecta uno diferente de la otra puede tener
implicancias sobre el impacto fiscal.

Encuentran que la reforma pospone el acceso a los beneficios
jubilatorios en 14 meses. Las mujeres y los trabajadores con menores
jubilaciones difieren en mayor medida.

La salida del mercado laboral se difiere en 10 meses. Este efecto es
mayor en los hombres y en trabajadores con jubilaciones mayores.

También encuentran que los trabajadores reaccionan poco a los cambios en
la SSW, y fuertemente a los ACCRUALS.

\hypertarget{austria}{%
\subsubsection{Austria}\label{austria}}

Este trabajo analiza una serie de cambios paraméticos en el sistema
jubilatorio austríaco en 2000 y 2004. Estos cambios implicaron un
aumento de la edad mínima de retiro (EMR). La introducción gradual de
este cambio permite identificar los efectos con precisión.

Se estima que el aumento en la ERA implicón un aumento de 0.4 años en la
salida del mercado laboral y 0.5 años en la edad de solicitud de la
jubilación.

La magnitud similar de este cambio implica que no se registraron
aumentos importantes en el uso de otros programas que pueden servir como
salidas alternativas del mercado de trabajo. El principal mecanismo de
ajuste es la mayor permanencia de los trabajadores en sus puestos de
trabajo.

\hypertarget{eua}{%
\subsubsection{EUA}\label{eua}}

\hypertarget{poluxedtica}{%
\subsubsection{Política}\label{poluxedtica}}

En 1983 se aprobó en EUA un aumento a la edad Normal de Retiro (NRA).
Desde el año 2000, la NRA subiría 2 meses por año. Los cambios en la
edad normal de retiro actuán como un ajuste actuarial. Cada aumento de
dos meses implica una pérdida de 1\% en las prestaciones. Estiman que el
50\% del aumento en la edad normal de retiro se traduce en aumentos de
la edad efectiva de retiro.

\hypertarget{reino-unido}{%
\subsection{Reino Unido}\label{reino-unido}}

This paper uses evidence on labor market behavior in the UK between 2010
and 2014 to examine what impact increasing the ERA from 60 to 62 has had
on the economic activity of the affected cohorts of women.

\hypertarget{literatura}{%
\subsubsection{Literatura}\label{literatura}}

Blundell and Emmerson (2007) estimate that a three-year increase in the
ERA for both men and women (and assuming that defined benefit
occupational pension schemes respond with a three-year increase in their
normal pension ages as well) would increase retirement ages by between
0.4 and 1.8 years, depending on the specification used.

However, while the effects estimated in these ex ante simulations were
quite large, if anything the results of ex post evaluations suggests
even larger effects.

Although almost all of these have focused on changes to normal, rather
than early, retirement ages. The two major exceptions are Staubli and
Zweimüller (2013) and Atalay and Barrett (2015), who examine the effect
of changes in ERAs. They find that a one year increase in the ERA led to
an increase in employment rates of 9.75 percentage points for affected
men and by 11 percentage points for affected women, with increases in
unemployment rates of a similar size.

Manoli and Weber (2016) study the same Austrian reforms and find large
delays in job exits and pension claiming caused by the increase in the
ERA. However, the Austrian state pension system is different from the UK
(and a number of other coun- tries' systems) in several important ways.
First, in the Austrian system, individuals' pension benefits are
completely withdrawn if their earnings exceed around \$500 a month.
Second, although the Austrian system provides some increase in pension
income for delayed drawing, this is done at a less than actuarially fair
rate.

Third, the Austrian state pension provides a very high level of earnings
replacement (according to Staubli and Zweimüller (2013), the average net
replacement rate of pre- retirement earnings is 75\%); public pensions,
therefore, provide the main source of income for most pensioners in
Austria.

Atalay and Barrett (2015) examine the effect of an increase in the
earliest age at which women can access the Australian Age Pension. They
find, using cross-sectional survey data, that a one year increase in the
eligibility age induced a 12--19 percentage point increase in fe- male
labor supply.

A number of papers have conducted ex ante simulation of such reforms
using out-of-sample predictions, which suggested quite large equilibrium
effects in many countries. Ex post evaluations of changes to normal and
early retirement ages have tended to find, if any- thing, larger effects
than were suggested by the ex ante simulations.

\hypertarget{la-poluxedtica}{%
\subsubsection{La política}\label{la-poluxedtica}}

En 1995, UK aprobó un aumento de la EDA para las mujeres. La ERA subiría
gradualmente de 60 a 65 años entre 2010 y 2020.

\hypertarget{los-resultados}{%
\subsubsection{Los resultados}\label{los-resultados}}

Identifican el impacto de la ERA comparando cohortes que enfrentan
diferentes ERAs.

Las tasas de empleo de las mujeres de 60 y 61 años aumentó 6.3 puntos
porcentuales cuando la ERA subió de 60 a 62 años. Este efecto es implica
un aumento de dos meses en la edad de retiro promedio, e implica que 3/4
de los retiros en exceso a los 60 se explican debido a la ERA.

Estos resultados también sugieren un fuerte efecto señalización de la
ERA. En UK hay incentivos financieros reducidos a retirarse en la ERA.
We find the policy has also led to a 1.2 percentage point increase in
the fraction of women who are unemployed and actively seeking work at
ages 60 and 61. We also find a 4.0 percentage point increase in the
proportion of women reporting themselves to be economically inactive due
to sickness or disability.

\hypertarget{noruega}{%
\subsection{Noruega}\label{noruega}}

\hypertarget{francia}{%
\subsection{Francia}\label{francia}}

Además de sus efectos en el empleo, el aumento de la edad de retiro
puede tener impactos en otros programas de seguridad social como el
seguro de desempleo, las pensiones por invalidez y el seguro de
enfermedad.

Estos efectos sustitución pueden ser de dos tipos. Por un lado, los
individuos que ya estan en estos programas prolongan su uso de los
mismos. Además, puede haber individuos que se encuentran trabajando
pueden usar estos programas como un puente para retirarse antes de la
nueva SEA.

Staubli and Zweimüller (2013) y Vestad (2013) son los únicos que toman
en cuenta este efecto. Las estimaciones difieren considerablemente, los
efectos de sustitución son iguales a los del empleo en Staubli and
Zweimüller (2013) y solo 1/3 en Vestad (2013).

\hypertarget{la-poluxedtica-1}{%
\subsubsection{La política}\label{la-poluxedtica-1}}

En 2010 la SEA subió de 60 a 62 años. En este paper solo se evalúa la
suba de 60 a 61.

The implementation of this reform is cohort-based: the parameters
gradually increase with the year of birth of the individual. Cohort 1951
is the first impacted by the reform, as the SEA increases to 60 years
and 4 months for individuals born in the second semester of the year.
The SEA then gradually increases and reaches 62 for cohort 1955, as
presented in Figure 1.

In this paper, we investigate these two potential effects by studying
the immediate impact of the 2010 reform of the French pension system on
the labor force participation of older workers. This reform increased
the SEA from 60 to 62, but we carry out a short-term evaluation focusing
on the 60--61 increase.

We expect the effect of the reform to be particularly strong for two
reasons. First, as the French pension system offers high replacement
rate at the SEA for individuals with a full career duration, the
pre-reform share of individuals retiring exactly at the SEA is high
compared with other countries, suggesting large employment effects of
its increase.

En Francia hay una alta proporción de los trabajadores que se retiran en
la SEA. La tasa de remplazo en la SEA es alta. Estos dos factores hacen
que el efecto esperado de la reforma sea alto.

Encuentran un alto efecto sobre el empleo 21 puntos porcentuales. 40\%
de la caída en la tasa de retiro. El resto termina en otros programas,
principalmente desempleo.

Se estiman los impactos de la reforma condicional al estatus anterior.
Los efectos sobre el empleo se concentran en trabajadores que se
encuentran empleados, y que los efectos sustitución son dominados por
los trabajadores fuera del mercado laboral.

Los efectos son mayores para individos con incentivos a retirarse
temprano y menores para individuos con mala salud.

Ver salidas y gráficos pueden ser interesantes.

\hypertarget{lecciones-de-poluxedtica}{%
\section{Lecciones de política}\label{lecciones-de-poluxedtica}}

\begin{itemize}
\tightlist
\item
  La edad mínima de jubilación es un instrumento altamente eficaz para
  prolongar la vida laboral de los mayores. Efecto señalización es parte
  del impacto.
\item
  Los incentivos actuariales también funcionan.
\item
  Jubilaciones mínimas pueden ser un incentivo a salir temprano del
  mercado a pesar de los ajustes actuariales.
\item
  Ojo con los otros programas.
\item
  Salud y educación, restricciones en las oportunidades de empleo.
\item
  Hacia donde va la agenda de investigación.
\end{itemize}

\hypertarget{glosario}{%
\section{Glosario}\label{glosario}}

\begin{description}
\item[Edad Efectiva de Retiro]
Es la edad a la que un individuo empieza a recibir prestaciones del
sistema de seguridad social.
\item[Edad Mínima de Retiro]
Es la menor edad a la que un trabajador puede aplicar a un programa de
seguridad social. Es usual que en estos casos las prestaciones se vean
reducidas frente a las recibidas en la edad normal de retiro.
\item[Retiro Temprano]
Es la práctica de acogerse a los beneficios de la seguirdad social antes
de la edad normal de retiro.
\item[Prueba de Ingresos]
Es un límite a los ingresos que puede tener alguien que recibe
prestaciones del sistema de segurida social.
\item[Impuestos Implícitos]
Es el impuesto implícito que enfrenta un trabajador cuando, pudiendo
recibir una prestación del sistema de seguirdad social, decide seguir
trabajando y sus beneficios futuros no son compensados.
\end{description}

\hypertarget{referencias}{%
\section*{Referencias}\label{referencias}}
\addcontentsline{toc}{section}{Referencias}

\hypertarget{refs}{}
\leavevmode\hypertarget{ref-cribb16}{}%
Cribb, J., C. Emmerson, and G. Tetlow. 2016. ``Signals Matter? Large
Retirement Responses to Limited Financial Incentives.'' \emph{Labour
Economics} 42: 203--12.

\leavevmode\hypertarget{ref-staubli13}{}%
Staubli, Stefan, and Josef Zweimüller. 2013. ``Does Raising the Early
Retirement Age Increase Employment of Older Workers?'' \emph{J. Public
Econ.} 108: 17--32.

\leavevmode\hypertarget{ref-vestad13}{}%
Vestad, Ola. 2013. ``Labour Supply Effects of Early Retirement
Provision.'' \emph{Labour Econ.}, no. 25: 98--109.

\end{document}
