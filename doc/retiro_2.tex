\PassOptionsToPackage{unicode=true}{hyperref} % options for packages loaded elsewhere
\PassOptionsToPackage{hyphens}{url}
%
\documentclass[]{article}
\usepackage{lmodern}
\usepackage{amssymb,amsmath}
\usepackage{ifxetex,ifluatex}
\usepackage{fixltx2e} % provides \textsubscript
\ifnum 0\ifxetex 1\fi\ifluatex 1\fi=0 % if pdftex
  \usepackage[T1]{fontenc}
  \usepackage[utf8]{inputenc}
  \usepackage{textcomp} % provides euro and other symbols
\else % if luatex or xelatex
  \usepackage{unicode-math}
  \defaultfontfeatures{Ligatures=TeX,Scale=MatchLowercase}
\fi
% use upquote if available, for straight quotes in verbatim environments
\IfFileExists{upquote.sty}{\usepackage{upquote}}{}
% use microtype if available
\IfFileExists{microtype.sty}{%
\usepackage[]{microtype}
\UseMicrotypeSet[protrusion]{basicmath} % disable protrusion for tt fonts
}{}
\IfFileExists{parskip.sty}{%
\usepackage{parskip}
}{% else
\setlength{\parindent}{0pt}
\setlength{\parskip}{6pt plus 2pt minus 1pt}
}
\usepackage{hyperref}
\hypersetup{
            pdftitle={Incentivos al Retiro},
            pdfborder={0 0 0},
            breaklinks=true}
\urlstyle{same}  % don't use monospace font for urls
\usepackage[margin=1in]{geometry}
\usepackage{graphicx,grffile}
\makeatletter
\def\maxwidth{\ifdim\Gin@nat@width>\linewidth\linewidth\else\Gin@nat@width\fi}
\def\maxheight{\ifdim\Gin@nat@height>\textheight\textheight\else\Gin@nat@height\fi}
\makeatother
% Scale images if necessary, so that they will not overflow the page
% margins by default, and it is still possible to overwrite the defaults
% using explicit options in \includegraphics[width, height, ...]{}
\setkeys{Gin}{width=\maxwidth,height=\maxheight,keepaspectratio}
\setlength{\emergencystretch}{3em}  % prevent overfull lines
\providecommand{\tightlist}{%
  \setlength{\itemsep}{0pt}\setlength{\parskip}{0pt}}
\setcounter{secnumdepth}{0}
% Redefines (sub)paragraphs to behave more like sections
\ifx\paragraph\undefined\else
\let\oldparagraph\paragraph
\renewcommand{\paragraph}[1]{\oldparagraph{#1}\mbox{}}
\fi
\ifx\subparagraph\undefined\else
\let\oldsubparagraph\subparagraph
\renewcommand{\subparagraph}[1]{\oldsubparagraph{#1}\mbox{}}
\fi

% set default figure placement to htbp
\makeatletter
\def\fps@figure{htbp}
\makeatother


\title{Incentivos al Retiro}
\author{}
\date{\vspace{-2.5em}}

\begin{document}
\maketitle

{
\setcounter{tocdepth}{3}
\tableofcontents
}
\hypertarget{introducciuxf3n}{%
\section{Introducción}\label{introducciuxf3n}}

\begin{itemize}
\tightlist
\item
  Importancia de alargar vida laboral. (OCDE 2011).
\item
  Seguridad Social es un factor determinante de la oferta laboral de los
  mayores.
\item
  Cambios en los sistemas previsionales en varios países para lograrlo.
\item
  Evidencia de los resultados de estas políticas.
\end{itemize}

\hypertarget{marco-conceptual}{%
\section{Marco Conceptual}\label{marco-conceptual}}

\hypertarget{impacto-de-la-seguridad-social-en-la-oferta-de-trabajo}{%
\subsection{Impacto de la Seguridad Social en la oferta de
trabajo}\label{impacto-de-la-seguridad-social-en-la-oferta-de-trabajo}}

El marco teórico mas usado para analizar el impacto de los parámetros de
los sistemas previsionales en las decisiones de retiro es un modelo en
el que el trabajador representativo maximiza la utilidad derivada del
consumo de bienes y de ocio.

La utilidad del trabajador depende positivamente de ambos. Si decide
trabajar más, puede consumir más pero se ve obligado a disfrutar de
menos ocio, por lo que existe un \emph{trade off} entre ambos bienes.

El sistema jubilatorio afecta esta decisión a través de dos canales: el
precio relativo de ambos bienes y la riqueza total del individuo. En
efecto, si el sistema penaliza más el trabajo, el ocio se vuelve más
barato relativo al consumo. Por otro lado, si las prestaciones del
sistema se reducen, el precio relativo de ambos bienes se mantiene pero
el trabajador va a acceder a un menor nivel de ambos.

Este modelo se puede extender para analizar fenómenos más complejos. En
primer lugar, el modelo supone que no hay restricciones de liquidez.
Esto implica que el trabajador tiene acceso al mercado financiero, lo
que le permite suavizar su consumo.

Si, por el contrario, el trabajador enfrenta restricciones de liquidez,
es posible que se vea obligado a seguir en el mercado de trabajo por más
tiempo para mantener sus niveles de consumo.

Por otro lado, el modelo no permite analizar decisiones conjuntas de
miembres de un mismo hogar. Algunos trabajos extienden este modelo para
analizar los impactos de los sistemas de seguridad social en los
hogares. En este marco, si el ocio de uno de los integrantes del hogar
es un bien complementario al del otro, los incentivos que enfrenta uno
de los integrantes del hogar afectan a otros. Esto permite explicar
como, por ejemplo, algunas parejas deciden retirarse juntas.

Finalmente otro fenómeno relevante es el de señalización. En este marco,
las edades de retiro dispuestas en las reglas de los sistemas
jubilatorios actúan como señales para que los trabajadores se retiren.

Si bien los resultados empíricos sobre esto no son concluyentes (ver
Hanel 2010), es posible que los trabajadores perciban las edades mínimas
y normal de retiro establecidas en el sistema como la edad a la que
\emph{deben} retirarse.

Finalmente, otro de los aspectos importantes a tener en cuenta en la
decisión de retiro es la existencia de oportunidades en el mercado de
trabajo para los trabajadores. Ver Hakola.

\hypertarget{paruxe1metros-de-la-seguridad-social}{%
\subsection{Parámetros de la Seguridad
Social}\label{paruxe1metros-de-la-seguridad-social}}

Uno de los hechos estilizados encontrado en la literatura es que una
gran proporción de los trabajadores se retira la EMR o la ENR. Citar a
({\textbf{???}}). Para Uruguay, ver ({\textbf{???}}). En este contexto,
aumentar estas edades surge como una medida con potencial de ser
altamente eficaz para prolongar la vida laboral de los trabajadores.

\hypertarget{edad-muxednima-de-retiro}{%
\subsubsection{Edad Mínima de Retiro}\label{edad-muxednima-de-retiro}}

Es la menor edad a la que un trabajador puede recibir una prestación del
sistema jubilatorio. Si bien este suele ser un parámetro muy saliente en
todo sistema previsional, es usual que los trabajadores cuenten con
otros vías para retirarse del mercado laboral antes de esta edad (seguro
de desempleo prolongado, , seguro de enfermedad, pensiones por
invalidez, etc.). En muchos países, estas vías terminan funcionando como
esquemas de retiro temprano.

\hypertarget{edad-normal-de-retiro}{%
\subsubsection{Edad Normal de Retiro}\label{edad-normal-de-retiro}}

La Edad Normal de Retiro es la menor edad a la que un trabajador puede
acceder a una prestación del sistema jubilatorio sin recibir
penalizaciones por retiro temprano.

\hypertarget{ajustes-actuariales}{%
\subsubsection{Ajustes Actuariales}\label{ajustes-actuariales}}

Los ajustes actuariales son deducciones que se le hacen a las
prestaciones que recibe el trabajador por acogerse a los beneficios
antes de la edad normal de retiro.

¿Son actuarialmente justos los ajustes?

\hypertarget{midiendo-los-incentivos-de-la-ss}{%
\subsection{Midiendo los incentivos de la
SS}\label{midiendo-los-incentivos-de-la-ss}}

\begin{itemize}
\tightlist
\item
  \$ SSW \$, Accrual, \(ITAX\).
\item
  Peak Value, Option Value
\end{itemize}

\hypertarget{literatura-empuxedrica}{%
\section{Literatura empírica}\label{literatura-empuxedrica}}

\hypertarget{primeros-trabajos}{%
\subsection{Primeros Trabajos}\label{primeros-trabajos}}

\begin{itemize}
\tightlist
\item
  Efectos de la \(SSW\) y del Accrual.
\end{itemize}

\hypertarget{poluxedticas-de-las-uxfaltimas-duxe9cadas}{%
\subsection{Políticas de las últimas
décadas}\label{poluxedticas-de-las-uxfaltimas-duxe9cadas}}

({\textbf{???}})

\hypertarget{resultados-de-estas-poluxedticas}{%
\subsection{Resultados de estas
políticas}\label{resultados-de-estas-poluxedticas}}

\hypertarget{ajustes-actuariales-y-edad-normal-de-retiro}{%
\subsubsection{Ajustes Actuariales y Edad Normal de
Retiro}\label{ajustes-actuariales-y-edad-normal-de-retiro}}

\hypertarget{edad-muxednima-de-retiro-1}{%
\subsubsection{Edad Mínima de Retiro}\label{edad-muxednima-de-retiro-1}}

\hypertarget{lecciones-para-uruguay}{%
\subsection{Lecciones para Uruguay}\label{lecciones-para-uruguay}}

\begin{itemize}
\tightlist
\item
  La edad mínima de jubilación es un instrumento altamente eficaz para
  prolongar la vida laboral de los mayores. Efecto señalización es parte
  del impacto.
\item
  Los incentivos actuariales también funcionan.
\item
  Jubilaciones mínimas pueden ser un incentivo a salir temprano del
  mercado a pesar de los ajustes actuariales.
\item
  Ojo con los otros programas.
\item
  Salud y educación, restricciones en las oportunidades de empleo.
\end{itemize}

\hypertarget{glosario}{%
\section{Glosario}\label{glosario}}

\begin{description}
\item[Edad Efectiva de Retiro]
Es la edad a la que un individuo empieza a recibir prestaciones del
sistema de seguridad social.
\item[Edad Mínima de Retiro]
Es la menor edad a la que un trabajador puede aplicar a un programa de
seguridad social. Es usual que en estos casos las prestaciones se vean
reducidas frente a las recibidas en la edad normal de retiro.
\item[Retiro Temprano]
Es la práctica de acogerse a los beneficios de la seguirdad social antes
de la edad normal de retiro.
\item[Prueba de Ingresos]
Es un límite a los ingresos que puede tener alguien que recibe
prestaciones del sistema de segurida social.
\item[Impuestos Implícitos]
Es el impuesto implícito que enfrenta un trabajador cuando, pudiendo
recibir una prestación del sistema de seguirdad social, decide seguir
trabajando y sus beneficios futuros no son compensados.
\end{description}

\hypertarget{referencias}{%
\section{Referencias}\label{referencias}}

\end{document}
