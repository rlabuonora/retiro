\PassOptionsToPackage{unicode=true}{hyperref} % options for packages loaded elsewhere
\PassOptionsToPackage{hyphens}{url}
%
\documentclass[]{article}
\usepackage{lmodern}
\usepackage{amssymb,amsmath}
\usepackage{ifxetex,ifluatex}
\usepackage{fixltx2e} % provides \textsubscript
\ifnum 0\ifxetex 1\fi\ifluatex 1\fi=0 % if pdftex
  \usepackage[T1]{fontenc}
  \usepackage[utf8]{inputenc}
  \usepackage{textcomp} % provides euro and other symbols
\else % if luatex or xelatex
  \usepackage{unicode-math}
  \defaultfontfeatures{Ligatures=TeX,Scale=MatchLowercase}
\fi
% use upquote if available, for straight quotes in verbatim environments
\IfFileExists{upquote.sty}{\usepackage{upquote}}{}
% use microtype if available
\IfFileExists{microtype.sty}{%
\usepackage[]{microtype}
\UseMicrotypeSet[protrusion]{basicmath} % disable protrusion for tt fonts
}{}
\IfFileExists{parskip.sty}{%
\usepackage{parskip}
}{% else
\setlength{\parindent}{0pt}
\setlength{\parskip}{6pt plus 2pt minus 1pt}
}
\usepackage{hyperref}
\hypersetup{
            pdftitle={Incentivos al Retiro},
            pdfborder={0 0 0},
            breaklinks=true}
\urlstyle{same}  % don't use monospace font for urls
\usepackage[margin=1in]{geometry}
\usepackage{graphicx,grffile}
\makeatletter
\def\maxwidth{\ifdim\Gin@nat@width>\linewidth\linewidth\else\Gin@nat@width\fi}
\def\maxheight{\ifdim\Gin@nat@height>\textheight\textheight\else\Gin@nat@height\fi}
\makeatother
% Scale images if necessary, so that they will not overflow the page
% margins by default, and it is still possible to overwrite the defaults
% using explicit options in \includegraphics[width, height, ...]{}
\setkeys{Gin}{width=\maxwidth,height=\maxheight,keepaspectratio}
\setlength{\emergencystretch}{3em}  % prevent overfull lines
\providecommand{\tightlist}{%
  \setlength{\itemsep}{0pt}\setlength{\parskip}{0pt}}
\setcounter{secnumdepth}{0}
% Redefines (sub)paragraphs to behave more like sections
\ifx\paragraph\undefined\else
\let\oldparagraph\paragraph
\renewcommand{\paragraph}[1]{\oldparagraph{#1}\mbox{}}
\fi
\ifx\subparagraph\undefined\else
\let\oldsubparagraph\subparagraph
\renewcommand{\subparagraph}[1]{\oldsubparagraph{#1}\mbox{}}
\fi

% set default figure placement to htbp
\makeatletter
\def\fps@figure{htbp}
\makeatother


\title{Incentivos al Retiro}
\author{}
\date{\vspace{-2.5em}}

\begin{document}
\maketitle

{
\setcounter{tocdepth}{3}
\tableofcontents
}
\hypertarget{introducciuxf3n}{%
\section{Introducción}\label{introducciuxf3n}}

\begin{itemize}
\tightlist
\item
  Importancia de alargar vida laboral. (OCDE 2011).
\end{itemize}

Atalay

Population aging poses an important challenge to the fiscal
sustainability of social security systems in many industrialized
economies. In addressing these challenges governments around the world
continue to implement reforms to their social security programs.
Restruc- turing the pension system, changing the level of benefit
payments, and tightening access such as by increasing the eligibility
age, are common examples of reforms that have recently been implemented.

\hypertarget{marco-conceptual}{%
\section{Marco Conceptual}\label{marco-conceptual}}

\hypertarget{impacto-de-la-seguridad-social-en-la-oferta-de-trabajo}{%
\subsection{Impacto de la Seguridad Social en la oferta de
trabajo}\label{impacto-de-la-seguridad-social-en-la-oferta-de-trabajo}}

Atalay

The theoretical literature on the incentive effects of social security
show that workers' retirement decisions are influenced through two main
channels. The first is by directly changing the life-time income or
expected wealth of an individual. If the program benefit exceeds the
individual's contribution to the program, existence of the program
increases the life-time income of the individual and therefore reduces
the labour supply of the individual on the assumption that leisure is a
normal good. This is known as the ``wealth effect'' of the program.

The second channel operates when social security benefit payments
increase with contributory earnings. In this case, an extra year of work
also increases the future stream of expected social security benefits.
When considering the optimal timing of retirement, workers will take
account of the effect of an extra year of work on the level of
retirement income when s/he eventually retires.

\begin{itemize}
\tightlist
\item
  Modelo de oferta de trabajo y retiro (Cribb, Emmerson, and Tetlow
  (2016)). Trade-off entre consumo y ocio.
\item
  Canales de impacto de la SS a la oferta laboral

  \begin{itemize}
  \tightlist
  \item
    Efecto sustitución (accrual)
  \item
    Efecto riqueza.
  \end{itemize}
\end{itemize}

Gruber and Wise (2004) and Samwick (1998) argue that the accrual effect
is the main driving source of retirement behavior in the reforms.

Cribb:

Gruber and Wise (2004) surveyed evidence on eleven developed countries
and highlighted the fact that labor force exits are concentrated around
legislated early and normal retirement ages and tend to be larg- er than
can be explained by the pure financial incentives associated with
retiring at these ages.

Most of the early papers that attempted to simulate the impact of moving
these early and normal retirement ages on labor force participation
relied on using out-of-sample predictions. Papers simulating changes in
early and normal retirement ages in the US suggested quite large effects
on retirement ages.

However, while the effects estimated in these ex ante simulations were
quite large, if anything the results of ex post evaluations suggests
even larger effects.

More recently, there have been a growing number of reforms around the
world, which have increased pension ages. Therefore, ex post evaluations
have become more common in the literature.

Women's economic activity could be affected by an increase in the ERA
through four main mechanisms. First, increasing the ER

A will have some effect on individuals' marginal financial incentives to
work, through changing marginal tax rates and eligibility for
out-of-work ben- efits. This channel will be significantly less
important in the UK than it is in some other countries because there is
no earnings test for state pen- sion receipt in the UK.

Second, the increase in the ERA reduces the length of time that indi-
viduals receive state pension income for and thus reduces their lifetime
wealth; this will tend to increase labor supply.

Third, individuals who are credit constrained may have to continue
working during the period when they are no longer able to receive their
state pension in order to finance their consumption.

Fourth, the ERA may provide a signal about the `appropriate' age at
which to retire

\begin{itemize}
\tightlist
\item
  Otros factores

  \begin{itemize}
  \tightlist
  \item
    Efecto señalización
  \item
    Restricciones de liquidez
  \item
    Oportunidades laborales
  \item
    Impacto sobre todo el hogar.
  \end{itemize}
\end{itemize}

\hypertarget{paruxe1metros-de-la-ss}{%
\subsection{Parámetros de la SS}\label{paruxe1metros-de-la-ss}}

\begin{itemize}
\tightlist
\item
  Edad Mínima de Retiro, Edad Normal de Retiro, Ajustes Actuariales.
\item
  Hay varios más en el glosario pero nos centramos en estos.
\end{itemize}

\hypertarget{midiendo-los-incentivos-de-la-ss}{%
\subsection{Midiendo los incentivos de la
SS}\label{midiendo-los-incentivos-de-la-ss}}

\begin{itemize}
\tightlist
\item
  \$ SSW \$, Accrual, \(ITAX\).
\end{itemize}

Para medir estos incentivos empíricamente, se calcula la riqueza de la
seguridad social (SSW). Esta medida compara los ingresos que el
trabajador recibe por trabajar con los cambios en el flujo de
prestaciones que le paga el sistema de jubilaciones.

La Riqueza de la Seguridad Social (SSW) para una edad de retiro \(h\) es
el valor actualizado de los beneficios recibidos entre \(h + 1\) y la
fecha de muerte \(S\).

\[ SSW_{h} = \sum_{s=h+1}^{\S} \rho_{s} B_{s}(h)\]

Para actualizar los beneficios se utiliza el factor de descuento
\(\rho\), que depende de la probabilidad de sobrevivencia y el factor de
descuento intertemporal.

Si seguir trabajando un año reduce el flujo de pagos que el individuo
recibe de la seguridad social, esta diferencia actúa como un impuesto a
seguir trabajando. Este desincentivo a seguir en la fuerza de trabajo
una vez alcanzadas las condiciones para acceder a una jubilación
funciona como un impuesto sobre el trabajo.

El devengamiento de \(SSW\) es la medida más simple de incentivo a
trabajar, consiste en calcular el cambio en SSW por permanecer en la
fuerza de trabajo un año más:

\[ SSA_{a} = SSW_{a+1} - SSW_{a}\]

El impuesto implícito a seguir trabajando se calcula como

\[ \tau_{a} = \frac{-SSA_{a}}{W_{a+1}}\]

El principal problema que tiene es que esta medida solo mira un período
para adelante, pero la SSA no se incrementa monotónicamente, sino que
puede tener saltos.

Eso implica que para un trabajador la SSA de un año puede ser baja, pero
la de 4 años para adelante tener un salto brusco, por lo que el
trabajador sigue trabajando teniendo ese salto en cuenta.

Para resolver esta limitación, se propone la medida de \emph{Peak
Value}, que calcula \(PV_{a} = max_{h}(SSW_{h}-SSW_{a})\), o sea, la
diferencia máxima entre la riqueza de la seguridad social de retirarse
hoy o retirarse en otra fecha. - Peak Value, Option Value

Consider that an individual takes the decision regarding the optimal
date of retirement entry according to the Option Value Model of Stock
and Wise (1990), i.e.~she compares all possible future streams of
utility from income and leisure and delays retirement entry if she can
thereby receive a higher stream of utility. In this framework, retiring
at any early date s instead of any later dater t has four effects: (i)
it lowers utility due to a loss of wage earnings during the period
between s and t, (ii) it increases utility from leisure during s and t,
(iii) it lengthens the period of benefit receipt, thereby raising
utility according to the amount of benefits that are paid between s and
t, and (iv) it changes the expected present value of future benefits
during the remaining lifetime according to the pension formula. Hence,
retiring at date s instead of t will be preferred only if the loss in
wage income is at least outweighed by (a) the utility from leisure
between s and t, (b) the retirement income received between s and t,
plus (b) the difference in the present values of future income if the
individual retires at date s instead of date t.

\hypertarget{literatura-empuxedrica}{%
\section{Literatura empírica}\label{literatura-empuxedrica}}

\hypertarget{primeros-trabajos}{%
\subsection{Primeros Trabajos}\label{primeros-trabajos}}

\begin{itemize}
\tightlist
\item
  Efectos de la \(SSW\) y del Accrual.
\end{itemize}

Hanel

One strand of the literature deals with the effect of expected income
from social security or pension benefits (i.e.~the level of social
security or pension wealth) on the retirement decision. Classical
life-cycle-models used for example by Gordon and Blinder (1980) or
Gustman and Steinmeier (1986) show that the amount of provided benefits
compared to potential wages has an influence on the retirement decision
of an individual maximizing utility from income and leisure.

Hurd (1990a) examined the peak in retirement entries in the United
States at age 62, when benefit receipt is first available. This peak has
grown over time with growing social security benefits. Blau (1994)
estimated hazard rates into retirement dependent on social security
wealth and found a strong connection between benefit levels and exit
rates. A well-known study to be mentioned is conducted by Krueger and
Pischke (1992).

There is a strong agreement across the literature that pension accruals
fundamentally influence the retirement decision. Samwick (1998) compared
the effect of the level of social security wealth and accruals in social
security wealth and found the impact of accruals to be the central
determinant of the timing of retirement.

\hypertarget{comentarios-y-problemas-sobre-estos-trabajos}{%
\subsubsection{Comentarios y Problemas sobre estos
trabajos}\label{comentarios-y-problemas-sobre-estos-trabajos}}

Atalay

The majority of the empirical research attempting to estimate the ef-
fect of social security incentives on retirement is based on cross
sectional variation. These studies, summarized in the detailed surveys
by Coile and Gruber (2007) and Chan and Stevens (2004), typically find
strong effects of social security incentives on retirement deci- sion. A
limitation with this approach is that since the social security policy
is the same for everyone at a point in time, identification may be
undermined by the correlation between program incentives and tastes for
retirement. Therefore it is very difficult to reliably disen- tangle the
effect of social security program parameters from differences in
preferences across individuals, or from general trends in retirement and
benefit levels over time.

Atalay

A well known example of this approach is Krueger and Pischke (1992) in
which they investigate a change to U.S. Social Security pro- visions in
1977. In contrast to many cross sectional studies, Krueger and Pischke
(1992) find a weak relationship between social security wealth and
labour supply.

\hypertarget{poluxedticas-de-las-uxfaltimas-duxe9cadas}{%
\subsection{Políticas de las últimas
décadas}\label{poluxedticas-de-las-uxfaltimas-duxe9cadas}}

({\textbf{???}})

Entre 1980 y 2018, la EMR subió en Bélgica, Dinamarca, Francia,
Alemania, Japón, Holanda, España, Suecia y el Reino Unido
(({\textbf{???}})). Como veremos más adelante, las variaciones de este
parámetro de política pueden usarse para estimar la respuesta en los
trabajadores con precisión.

La ENR también subió generalizadamente en los países de la OCDE.

Los ajustes actuariales penalizan a los trabajadores que se retiran
antes de la ENR. Dado que al postergar un año el retiro, el trabajador
contribuye un año más a la seguridad social y recibe un año menos de
prestaciones, es usual que el valor de estas prestaciones sea mayor
cuanto más postergue el trabajaro el acceso a las mismas.

\hypertarget{resultados-de-estas-poluxedticas}{%
\subsection{Resultados de estas
políticas}\label{resultados-de-estas-poluxedticas}}

Atalay

investigate two issues: (i) to what extent this policy reform
contributed to an increase in the labour force participation of women,
and (ii) the degree to which the reform had an unintended side-effect of
inducing participation in alternative government programs, especially
the Disability Support Pension.

The key challenge in the empirical literature is to find a substantial
and plausibly exoge- nous variation in the social security system to
identify and gauge the behavioural impacts of public pensions.

\hypertarget{ajustes-actuariales-y-edad-normal-de-retiro}{%
\subsubsection{Ajustes Actuariales y Edad Normal de
Retiro}\label{ajustes-actuariales-y-edad-normal-de-retiro}}

\hypertarget{edad-muxednima-de-retiro}{%
\subsubsection{Edad Mínima de Retiro}\label{edad-muxednima-de-retiro}}

\hypertarget{lecciones-para-uruguay}{%
\subsection{Lecciones para Uruguay}\label{lecciones-para-uruguay}}

\begin{itemize}
\tightlist
\item
  La edad mínima de jubilación es un instrumento altamente eficaz para
  prolongar la vida laboral de los mayores. Efecto señalización es parte
  del impacto.
\item
  Los incentivos actuariales también funcionan.
\item
  Jubilaciones mínimas pueden ser un incentivo a salir temprano del
  mercado a pesar de los ajustes actuariales.
\item
  Ojo con los otros programas.
\item
  Salud y educación, restricciones en las oportunidades de empleo.
\end{itemize}

\hypertarget{glosario}{%
\section{Glosario}\label{glosario}}

\begin{description}
\item[Edad Efectiva de Retiro]
Es la edad a la que un individuo empieza a recibir prestaciones del
sistema de seguridad social.
\item[Edad Mínima de Retiro]
Es la menor edad a la que un trabajador puede aplicar a un programa de
seguridad social. Es usual que en estos casos las prestaciones se vean
reducidas frente a las recibidas en la edad normal de retiro.
\item[Retiro Temprano]
Es la práctica de acogerse a los beneficios de la seguirdad social antes
de la edad normal de retiro.
\item[Prueba de Ingresos]
Es un límite a los ingresos que puede tener alguien que recibe
prestaciones del sistema de segurida social.
\item[Impuestos Implícitos]
Es el impuesto implícito que enfrenta un trabajador cuando, pudiendo
recibir una prestación del sistema de seguirdad social, decide seguir
trabajando y sus beneficios futuros no son compensados.
\end{description}

\hypertarget{referencias}{%
\section*{Referencias}\label{referencias}}
\addcontentsline{toc}{section}{Referencias}

\hypertarget{refs}{}
\leavevmode\hypertarget{ref-cribb16}{}%
Cribb, J., C. Emmerson, and G. Tetlow. 2016. ``Signals Matter? Large
Retirement Responses to Limited Financial Incentives.'' \emph{Labour
Economics} 42: 203--12.

\end{document}
